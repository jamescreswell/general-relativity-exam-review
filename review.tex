\documentclass[11pt]{article}

\usepackage{fullpage}
\usepackage{parskip}
\usepackage{amsmath}
\usepackage{physics}

\begin{document}

\section{Equivalence principle and geodesics}

The \textbf{Einstein equilvalence principle} (EEP) is inspired by the weak equivalence principle (WEP), which follows from the universal behaviour of gravity.
Galileo and Newton observed that the inertial mass $m_i$ appearing in Newton's second law,
$$ \boldsymbol{F} = m_i \boldsymbol{a}, $$
is equal to the gravitational mass appearing in the gravitational force law,
$$ \boldsymbol{F} = -m_g \nabla \Phi. $$
Hence the WEP concludes that every object that is unaccelerated (i.e. freely falling, or subject only to gravity) moves along the same trajectory, regardless of its mass or composition. 
As a consequence, the motion of freely falling particles is the same in a gravitational field and a uniformly accelerated frame, in small enough regions of spacetime.
The EEP is an extension of the WEP, in which all physical measurements are included, not just the motion of particles.
Namely, the EEP states that in small enough regions of sapcetime, the laws of physics reduce to those of special relativity; it is impossible to detect the existence of a gravitational field by means of local experiments.
The EEP implies that there is no ``absolute acceleration''.
Instead we define ``unaccelerated'' as ``freely falling in the local gravitational field''.
Gravity is not a force because it doesn't cause acceleration (``acceleration due to gravity'' doesn't make sense).
This suggests that gravity can be attributed to the curvature of spacetime.

The EEP can be used to derive some basic quantitative results.
First is the \textbf{gravitational redshift}.
Consider two rockets, a distance $z$ apart, each moving with some constant acceleration $a$ in a certain direction, in a region away from any gravitational fields.
The trailing rocket emits a photon with wavelength $\lambda_0$, which arrives at the leading rocket after a time $\Delta t = z/c$ in our frame.
In this time the speed of the rockets has increased by $a \Delta t = az/c$.
Therefore there is a normal Doppler shift due to the difference in speed between the emitter and the receiver of the photon:
$$\frac{\Delta \lambda}{\lambda_0} = \frac{\Delta v}{c} =\frac{az}{c^2}.$$
$\Delta \lambda$ is positive so the photon is redshifted.
According to the EEP, this situation is indistinguishable from the sending of a photon upwards in a uniform downwards gravitational field with magnitude $a$, i.e.
$$ g = -\nabla \Phi = -a .$$
There is no way for the emitter and receiver to decide whether they are in accelerating rockets or in a gravitational field.
Therefore they also observe redshift, which can be expressed in terms of the gravitational potential:
$$ \frac{\Delta \lambda}{\lambda_0} = \frac{1}{c^2} \frac{\partial \Phi}{\partial z} z = \frac{\Delta \Phi}{c^2} $$
where the final equality follows because $\Phi$ is linear for a constant gravitational field.
Note that in the gravitational field of a planet, $\Phi = -G M/r$ increases as $r$ increases, so $\Delta \Phi$ and hence $\Delta \lambda$ are positive for photons travelling upwards.

Another implication of the EEP is \textbf{time dilation in a gravitational field}.
Consider the experimental setup above.
The period of the light when emitted is $T_0 = \lambda_0/c$, and the period when received is $T_1 = (\lambda_0 + \Delta \lambda)/c$.
Therefore, higher in the gravitational field, time runs faster.
In terms of $\Phi$,
$$ \frac{T_1}{T_0} = 1 + \frac{\Delta \Phi}{c^2}. $$
Calibration for this effect is crucial for GPS satelites.

These considerations imply the \textbf{need for a general metric}.

\section{Principle of general covariance}

\end{document}